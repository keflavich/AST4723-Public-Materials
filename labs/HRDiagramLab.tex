\documentclass[11pt]{article}
%\usepackage{geometry}
\usepackage[inner=1.5cm,outer=1.5cm,top=2.5cm,bottom=2.5cm]{geometry}
\pagestyle{empty}
\usepackage{graphicx}
\usepackage{fancyhdr, lastpage, bbding, pmboxdraw}
\usepackage[usenames,dvipsnames]{color}
\usepackage{indentfirst}
\definecolor{darkblue}{rgb}{0,0,.6}
\definecolor{darkred}{rgb}{.7,0,0}
\definecolor{darkgreen}{rgb}{0,.6,0}
\definecolor{red}{rgb}{.98,0,0}
\usepackage[colorlinks,pagebackref,pdfusetitle,urlcolor=darkblue,citecolor=darkblue,linkcolor=darkred,bookmarksnumbered,plainpages=false,pdflang=en-US,pdftitle=]{hyperref}
\renewcommand{\thefootnote}{\fnsymbol{footnote}}

\pagestyle{fancyplain}
\fancyhf{}
\lhead{ \fancyplain{}{AST4723} }
%\chead{ \fancyplain{}{} }
\rhead{ \fancyplain{}{\today} }
%\rfoot{\fancyplain{}{page \thepage\ of \pageref{LastPage}}}
\fancyfoot[RO, LE] {page \thepage\ of \pageref{LastPage} }
\thispagestyle{plain}

%%%%%%%%%%%% LISTING %%%
\usepackage{listings}
\usepackage{caption}
\DeclareCaptionFont{white}{\color{white}}
\DeclareCaptionFormat{listing}{\colorbox{gray}{\parbox{\textwidth}{#1#2#3}}}
\captionsetup[lstlisting]{format=listing,labelfont=white,textfont=white}
\usepackage{verbatim} % used to display code
\usepackage{fancyvrb}
\usepackage{acronym}
\usepackage{amsthm}
\VerbatimFootnotes % Required, otherwise verbatim does not work in footnotes!

\def\todo#1{\textcolor{red}{#1}}


\definecolor{OliveGreen}{cmyk}{0.64,0,0.95,0.40}
\definecolor{CadetBlue}{cmyk}{0.62,0.57,0.23,0}
\definecolor{lightlightgray}{gray}{0.93}



\lstset{
%language=bash,                          % Code langugage
basicstyle=\ttfamily,                   % Code font, Examples: \footnotesize, \ttfamily
keywordstyle=\color{OliveGreen},        % Keywords font ('*' = uppercase)
commentstyle=\color{gray},              % Comments font
numbers=left,                           % Line nums position
numberstyle=\tiny,                      % Line-numbers fonts
stepnumber=1,                           % Step between two line-numbers
numbersep=5pt,                          % How far are line-numbers from code
backgroundcolor=\color{lightlightgray}, % Choose background color
frame=none,                             % A frame around the code
tabsize=2,                              % Default tab size
captionpos=t,                           % Caption-position = bottom
breaklines=true,                        % Automatic line breaking?
breakatwhitespace=false,                % Automatic breaks only at whitespace?
showspaces=false,                       % Dont make spaces visible
showtabs=false,                         % Dont make tabls visible
columns=flexible,                       % Column format
morekeywords={__global__, __device__},  % CUDA specific keywords
}

\begin{document}

\title{HR Diagram Final Project}
% not used in 2022

\section{Goals}
In this project, you will try to measure the age of the cluster, the distance
to the cluster, and determine the spectral type of the most massive star you
observed.

\section{Background}
In the photometry lab, you reduced several images by:
\begin{itemize}
    \item dark-subtracting
    \item flat-fielding
    \item aligning and stacking
    \item cataloging the stars
    \item cross-matching the catalogs
    \item calculating errors
\end{itemize}

You will now make physical measurements from these data.


\section{Calibration}

Determining the age requires using color-magnitude diagrams, which you have already made.
However, the CMDs you created were not calibrated.  You need to calibrate the data!
Also, your catalogs from the assignment may not have been complete - re-tune the DAOPhot
parameters until you're able to extract all of the stars you can see by eye.
\textbf{HINT:} What is the FWHM of a star in the image?  What FWHM did you use in the assignment?


To calibrate the data, you must identify a non-variable star with well-measured flux
in a given band.  You must calibrate \emph{each} band, and you might want to select
different stars in each band!

\subsection{Identification / Finder Charts}
For each of these stars, you need to identify them: figure out exactly what their
RA/Dec coordinates are and/or determine their names.  You do this by comparing to
a finder chart.  Finder chart making is described in the first Observation Planning
lecture from AST 3722 (\url{https://ufl.instructure.com/courses/404100/pages/ast3722-resources}).

Once you've identified the star, look up its magnitude measurements
in the observed bands (UBVRI, H$\alpha$, S II $\lambda\lambda6716,6731$, O III 5007).
It is likely that there will be no narrow-band measurements available, though - save those for last,
as we can take another approach for them.
You can look up photometric magnitudes of stars with Vizier or SIMBAD.  If you get magnitudes from
SIMBAD, be sure to reference the original source!


Use a published catalog magnitude to calibrate your DAOPhot catalog.
The calibration factor is simply the magnitude you measured in the catalog $m_{measured}$ 
minus the published magnitude $m_{published}$, i.e., $m_{calfactor} = m_{measured} - m_{published}$.
You then subtract this calibration factor from the rest of the sources in the catalog.
What uncertainty does this add to your measurements?

\textbf{Check your calibration.}  Once you've applied the calibration to your catalog, you can check
that other stars have correct magnitudes according to the published catalog.  If you see a systematic difference,
i.e., multiple stars have wrong magnitude, your calibration factor may be incorrect!
What could cause this?  If you identify this problem, correct it!

\section{Calibrated Color-Magnitude and Color-Color Diagrams}

Once you have calibrated your data, make color-color and color-magnitude diagrams.
Of course, to do this, you need cross-matched catalogs - i.e., you need to have photometry in multiple bands
for each star.

\emph{Bonus points} for including appropriate statistical errors on the photometric data points and in the plots.
This is assigned as \emph{bonus} because we've never worked an example completely.  You have the appropriate
tools in hand: you have the error image, you know how to propagate errors, but you don't necessarily know
how the fluxes (and magnitudes) in the \texttt{DAOPhot} catalog were derived.
To figure that out, you have to read up on the algorithm here: \url{https://ui.adsabs.harvard.edu/abs/1987PASP...99..191S/abstract},
which is a bit thick.  In brief, \texttt{DAOPhot} convolves the image with a modified version of a Gaussian kernel in which the `wings'
are negative, then reports the local peak of the convolved image as the flux.  The error on this estimate is nontrivial.
A more reasonable approach might be to do formal aperture
(\url{https://photutils.readthedocs.io/en/stable/getting_started.html}) or PSF
photometry and use those measurements to obtain error estimates.
These approaches are well-described in \url{https://photutils.readthedocs.io/en/stable/aperture.html#error-estimation},
and less well-described (but still available) in \url{https://photutils.readthedocs.io/en/stable/psf.html}.
Aperture photometry is basically adding up pixels, so it's conceptually simple.  PSF photometry is fitting models to the data,
so we end up using the covariance matrix from the fit for the uncertainty.


\section{Measurements: Isochrone Matching}

The key aims, once you have calibrated CMDs and CCDs, are:
\begin{enumerate}
    \item Determine the distance to the cluster
    \item Determine the age of the cluster
    \item Determine the spectral type of the most massive star in the cluster
\end{enumerate}

Both items 1 and 2 above come from comparing the cluster CMD to a variety of isochrones.
You will need to obtain isochrones from \url{http://stev.oapd.inaf.it/cgi-bin/cmd}.
This page has a lot of options, but do not be intimidated!  You only need to modify the
\textbf{ages} (you want each of your isochrones to have only one age) and, maybe,
the \textbf{interstellar extinction}.  You can leave the rest untouched!
You may need to experiment with a range of different ages and extinction values;
use information from your astrophysics classes to figure out what ages might
be relevant to an open cluster.  I recommend varying ages, but keeping extinction
fixed, until you've come up with a reasonable age estimate.

Compare the isochrones to your observed CMD by overplotting them.
Use the vertical offset to determine the \emph{distance modulus} to the cluster.
Make sure your distance is reasonable!  This is a star cluster, so it's within our galaxy,
and it's more distant than the closest stars.

Once you have found a reasonable estimate for the distance, try to fine-tune your selection
of age and extinction.  Plot the theoretical isochrones with magnitudes correct for distance
using your inferred distance modulus.  Note that if you change the extinction value, the distance you infer will
also change.  In the Milky Way, extinction is usually $\approx A_V=2$ magnitudes per kiloparsec,
so be wary if you get an answer that is too far different from that (for example, if you found
$A_V=5$ and a distance of 1 pc, that would be a red flag that something is wrong!).

When trying to estimate the extinction, it may be helpful to use the color-color diagrams.
Exinction changes the colors of stars more subtly than it changes their total light, so you
may be able to determine the extinction just by looking at the position of the isochrone
vs the position of your data in the color-color diagram.

Determine the best-fit age.  Compare to isochrones from similar-age clusters.  By how much
do you have to change the age before you can tell it doesn't fit?  Report this as your age uncertainty.

Similarly, compare different distance modulus estimates to determine the uncertainty on your distance
modulus.  How accurately can you determine the distance to the cluster?  Report this as your distance uncertainty.

Finally, do the same for extinction estimates.

When you're deriving the values above, you will probably need to make many different plots.
However, what you turn in should be a set of color-magnitude and/or color-color diagrams
with your best-fit isochrone overlaid, along with the closest two isochrones
with lower/higher age/extinction/distance that you can distinguish.


Note that these estimates are not formal propagation-of-error derived estimates!  It is possible to achieve
those, but it requires modeling the cluster isochrone more directly.


Finally, report the spectral type (e.g., B6V, O2III) of the most massive star in the cluster, or report its mass.
You should be able to determine this approximately by comparing to the isochrones.
This is an important sanity check!  If you found an O3 star in a cluster that is 5 billion years old, you would say
something is probably wrong - O-stars die in only a few million years!
Report the uncertainty on the spectral type determined any way you like, just explain it.
Color-color diagrams and narrow-band filters may be particularly helpful for this measurement.

\section{What are you turning in?}

Write this up as a formal writeup with introduction, methods, results/analysis, and conclusion.
Include a table showing all of the catalogued measurements.  If you can't fit
everything onto one page, show a sample table in the writeup and turn in a
complete digital table in a machine-readable format.

You will be graded on:
\begin{itemize}
    \item References - did you describe where you got your finder chart and catalog measurements?
    \item Format - did you include relevant sections?  Did you produce a table (see below)?
    \item Methods - did you clearly report your methods in a way that can be easily reproduced?
        Did you document each step appropriately?
    \item Errors - Did you report appropriate statistical errors on your measurements?
        Did you address possible non-statistical sources of error?
    \item Age, distance, extinction - Did you report measurements of the age, distance, and extinction
        toward the cluster?
    \item Plots - did you show appropriate plots?  Did you label them adequately?
    \item Tables - did you include a photometry table?  Did it include all of the bands?
    \item Upper Limits and Completeness - Discuss what the faintest star is that you detected in each
        band.  What band was the most sensitive?  Bonus if you formally estimate the completeness, but
        we likely will not discuss this in class.
    \item Meeting general scientific and laboratory norms - Were items that needed labeling labeled?
        Were appropriate significant figures used? etc
\end{itemize}



\end{document}
% resource: https://www.youtube.com/watch?v=_eMNRa-KEiQ
