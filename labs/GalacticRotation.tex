\documentclass[11pt]{article}
%\usepackage{geometry}
\usepackage[inner=1.5cm,outer=1.5cm,top=2.5cm,bottom=2.5cm]{geometry}
\pagestyle{empty}
\usepackage{graphicx}
\usepackage{fancyhdr, lastpage, bbding, pmboxdraw}
\usepackage[usenames,dvipsnames]{color}
\usepackage{indentfirst}
\definecolor{darkblue}{rgb}{0,0,.6}
\definecolor{darkred}{rgb}{.7,0,0}
\definecolor{darkgreen}{rgb}{0,.6,0}
\definecolor{red}{rgb}{.98,0,0}
\usepackage[colorlinks,pagebackref,pdfusetitle,urlcolor=darkblue,citecolor=darkblue,linkcolor=darkred,bookmarksnumbered,plainpages=false,pdflang=en-US,pdftitle=]{hyperref}
\renewcommand{\thefootnote}{\fnsymbol{footnote}}

\pagestyle{fancyplain}
\fancyhf{}
\lhead{ \fancyplain{}{AST4723} }
%\chead{ \fancyplain{}{} }
\rhead{ \fancyplain{}{\today} }
%\rfoot{\fancyplain{}{page \thepage\ of \pageref{LastPage}}}
\fancyfoot[RO, LE] {page \thepage\ of \pageref{LastPage} }
\thispagestyle{plain}

%%%%%%%%%%%% LISTING %%%
\usepackage{listings}
\usepackage{caption}
\DeclareCaptionFont{white}{\color{white}}
\DeclareCaptionFormat{listing}{\colorbox{gray}{\parbox{\textwidth}{#1#2#3}}}
\captionsetup[lstlisting]{format=listing,labelfont=white,textfont=white}
\usepackage{verbatim} % used to display code
\usepackage{fancyvrb}
\usepackage{acronym}
\usepackage{amsthm}
\VerbatimFootnotes % Required, otherwise verbatim does not work in footnotes!

\def\todo#1{\textcolor{red}{#1}}


\definecolor{OliveGreen}{cmyk}{0.64,0,0.95,0.40}
\definecolor{CadetBlue}{cmyk}{0.62,0.57,0.23,0}
\definecolor{lightlightgray}{gray}{0.93}



\lstset{
%language=bash,                          % Code langugage
basicstyle=\ttfamily,                   % Code font, Examples: \footnotesize, \ttfamily
keywordstyle=\color{OliveGreen},        % Keywords font ('*' = uppercase)
commentstyle=\color{gray},              % Comments font
numbers=left,                           % Line nums position
numberstyle=\tiny,                      % Line-numbers fonts
stepnumber=1,                           % Step between two line-numbers
numbersep=5pt,                          % How far are line-numbers from code
backgroundcolor=\color{lightlightgray}, % Choose background color
frame=none,                             % A frame around the code
tabsize=2,                              % Default tab size
captionpos=t,                           % Caption-position = bottom
breaklines=true,                        % Automatic line breaking?
breakatwhitespace=false,                % Automatic breaks only at whitespace?
showspaces=false,                       % Dont make spaces visible
showtabs=false,                         % Dont make tabls visible
columns=flexible,                       % Column format
morekeywords={__global__, __device__},  % CUDA specific keywords
}

\begin{document}

\title{Galactic Rotation}
2020 version; not used in 2022

\section{Goal: Measure the Galactic Rotation Curve}
In the Radio Astronomy lab, we took a bunch of observations of the Galaxy, fitted spectral lines, and tried to estimate masses.  But we came up with masses that were... not quite right.
This is because most, or all, of what we saw was \emph{local} HI - atomic hydrogen that is really close to the sun, and therefore doesn't move much with respect to the sun.  It can't be used
to trace Galactic rotation.

However, we did detect some gas that \emph{does} follow the Galactic rotation curve.

\section{Overview}
In this project, you will try to detect that gas and measure the rotation curve correctly!

You will use all of the data the class obtained in the Radio Astronomy lab.
These will be distributed with an associated README file.

Your tasks are to:

\begin{enumerate}
    \item Map out the locations that were observed on a map of the Galaxy.  Use
        this map to determine which observations might be useful for assessing
        Galactic rotation.
    \item Identify any observations that have bad metadata - were some observed
        at the wrong time?  Do some have mis-reported alt/az?  Determine from
        the READMEs whether they can be corrected - correct those that can be,
        remove the rest.
    \item Identify at least six spectra obtained along different parts of the
        Galactic rotation curve.  Be sure to write up how you picked these -
        how did you decide that these spectra are on different parts of the
        Galactic rotation curve?  Use these for further analysis.
    \item Check for redundant spectra (spectra taken of the same location) from
        different observers.  Note if they are consistent.  If they are not consistent,
        go back and see if perhaps one has a mis-reported alt/az or some other error.
    \item Measure the \emph{local} velocity (the brightest peak) and all
        detectable sub-peaks in the spectra.
        This is probably the hard part of the project.
    \item Plot the fitted velocities versus their Galactic radii assuming the
        highest velocity along each line of sight comes from the Galactic
        tangent point.
\end{enumerate}

\section{Details}
Most of the work above is something you've done or been shown in previous lectures.

You will need Lectures 7 and 8 for the first three items above.
\url{https://ufl.instructure.com/courses/404100/files/52976106/download?wrap=1} and \url{https://ufl.instructure.com/courses/404100/files/53223060/download?wrap=1}

To measure the velocity, you will be performing multi-component velocity
fitting.  Note that you may need to use the high-frequency or low-frequency
part of the frequency-switched spectrum to measure the right component,
otherwise the `negative' spectrum might mess with the measurement.  Recall that
the frequency throw is 100 km/s in the frequency-switched measurements: that
means if there is a detectable signal sitting at -100 km/s (higher frequency =
blueshifted), the second observation will overlap with the first!

\subsection{Fitting}
The fitting approach you take is up to you, but the simplest is likely a
`peeling' approach, in which you first get the best fit you can to the main
peak, subtract that off, then try to fit the next-biggest peak in the leftovers
(the residual).

This is tricky!  You must make sure not to create artificial spectral features
by subtracting an incorrect model!

\subsection{Velocities}
You will convert your fitted spectral peak values into line-of-sight
velocities.  This will require correcting from topocentric (TOPO) velocity
to LSRK velocity as in the \texttt{Velocity and Galactic Rotation Exercise} notebook.


\section{What are you turning in?}

Write this up as a formal writeup with introduction, methods, results/analysis, and conclusion.
Include a table showing all of the velocity meausurements in both rest-frame and LSRK coordinates
along with measurement errors.

You will be graded on:
\begin{itemize}
    \item Data Selection - Did you plot up the collected data appropriately?  Did you show
        where the spectra were obtained, and determine which were usable to measure a rotation curve?
        Did you identify and reject bad data?
    \item Methods - did you clearly report your fitting approach?  Did you show the
        spectra with best-fit models overlaid?  Did you adjust from TOPO to LSR?
        Did you determine the Galactocentric radius appropriate for your measurements?
    \item Errors - Did you report appropriate statistical errors on your measurements?
        Did you address possible non-statistical sources of error?
    \item Plots - did you show appropriate plots?  Did you label them adequately?
    \item Tables - did you include a photometry table?  Did it include all of the bands?
    \item Format - did you include relevant sections? 
    \item References - Did you reference any works?   If so, did you cite them?
    \item Meeting general scientific and laboratory norms - Were items that needed labeling labeled?
        Were appropriate significant figures used? etc
\end{itemize}



\end{document}
% resource: https://www.youtube.com/watch?v=_eMNRa-KEiQ
