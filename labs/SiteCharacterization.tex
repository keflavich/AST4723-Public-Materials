\documentclass[11pt]{article}
%\usepackage{geometry}
\usepackage[inner=1.5cm,outer=1.5cm,top=2.5cm,bottom=2.5cm]{geometry}
\pagestyle{empty}
\usepackage{graphicx}
\usepackage{amsmath}
\usepackage{amssymb}
\usepackage{fancyhdr, lastpage, bbding, pmboxdraw}
\usepackage[usenames,dvipsnames]{color}
\definecolor{darkblue}{rgb}{0,0,.6}
\definecolor{darkred}{rgb}{.7,0,0}
\definecolor{darkgreen}{rgb}{0,.6,0}
\definecolor{red}{rgb}{.98,0,0}
\usepackage[colorlinks,pagebackref,pdfusetitle,urlcolor=darkblue,citecolor=darkblue,linkcolor=darkred,bookmarksnumbered,plainpages=false,pdflang=en-US,pdftitle=AST4723_SiteCharacterization_2022]{hyperref}
\renewcommand{\thefootnote}{\fnsymbol{footnote}}

\pagestyle{fancyplain}
\fancyhf{}
\lhead{ \fancyplain{}{AST4723} }
%\chead{ \fancyplain{}{} }
\rhead{ \fancyplain{}{\today} }
%\rfoot{\fancyplain{}{page \thepage\ of \pageref{LastPage}}}
\fancyfoot[RO, LE] {page \thepage\ of \pageref{LastPage} }
\thispagestyle{plain}

%%%%%%%%%%%% LISTING %%%
\usepackage{listings}
\usepackage{caption}
\DeclareCaptionFont{white}{\color{white}}
\DeclareCaptionFormat{listing}{\colorbox{gray}{\parbox{\textwidth}{#1#2#3}}}
\captionsetup[lstlisting]{format=listing,labelfont=white,textfont=white}
\usepackage{verbatim} % used to display code
\usepackage{fancyvrb}
\usepackage{acronym}
\usepackage{amsthm}
\VerbatimFootnotes % Required, otherwise verbatim does not work in footnotes!

\def\todo#1{\textcolor{red}{#1}}


\definecolor{OliveGreen}{cmyk}{0.64,0,0.95,0.40}
\definecolor{CadetBlue}{cmyk}{0.62,0.57,0.23,0}
\definecolor{lightlightgray}{gray}{0.93}



\lstset{
%language=bash,                          % Code langugage
basicstyle=\ttfamily,                   % Code font, Examples: \footnotesize, \ttfamily
keywordstyle=\color{OliveGreen},        % Keywords font ('*' = uppercase)
commentstyle=\color{gray},              % Comments font
numbers=left,                           % Line nums position
numberstyle=\tiny,                      % Line-numbers fonts
stepnumber=1,                           % Step between two line-numbers
numbersep=5pt,                          % How far are line-numbers from code
backgroundcolor=\color{lightlightgray}, % Choose background color
frame=none,                             % A frame around the code
tabsize=2,                              % Default tab size
captionpos=t,                           % Caption-position = bottom
breaklines=true,                        % Automatic line breaking?
breakatwhitespace=false,                % Automatic breaks only at whitespace?
showspaces=false,                       % Dont make spaces visible
showtabs=false,                         % Dont make tabls visible
columns=flexible,                       % Column format
morekeywords={__global__, __device__},  % CUDA specific keywords
}


\begin{document}
\title{Site Characterization lab: What can you see, and when?}
\maketitle

\section{Overview}

Site characterization is an important part of modern astronomy.  Before we
build a telescope, we need to know how it will perform!  Sites have many
important characteristics: longitude and latitude, elevation, weather
statistics, astronomy-specific weather statistics, accessibility,
infrastructure, available area, cost, stability (earthquakes?), and even
politics.

In this lab, you will characterize two observing sites:
\begin{enumerate}
\item Your home (wherever you're living during this class)
    [this must be completed \emph{before} you attempt the Radio Astronomy lab]
\item The front of the Bryant Space Science Center
    (this can be done when you're setting up for the Radio Astronomy lab)
\end{enumerate}

For an optical site characterization, you would usually use a tool called a 
Differential Image Motion Monitor (DIMM) to measure the seeing (blurring
by the atmosphere) from night-to-night.  In this lab, instead, you'll
be characterizing the site for radio astronomy.

Write up this lab as a detailed report.  Answer all of the questions
and include pictures and figures to show needed details.
No error analysis is needed, but you should include at least a brief
introduction and conclusion.  In your conclusion, evaluate how good
a site this is for radio astronomy.


\section{Background: Your planned observatory}
You are planning for a future lab in which you will use a radio
telescope, which is about 1.5m tall and 1.5m wide, to observe the Galaxy.

The telescope stands on a narrow tripod and needs to rest on flat ground.

The site needs to have low foot-traffic and no vehicle traffic so you don't
knock over the telescope.

The telescope will be operated from a laptop with a 10 foot coaxial cable.  You
need to allow for some slack, so you have to have space to set up a table and
chair.  You also need to power the laptop, so you need access to an electrical
outlet (infrastructure).

\subsection{Describe the Site}

Where will you do your observing?  Where can you put a telescope that is about 1.5m tall and 1.5m wide?
You need a flat area with a clear view of the sky.

Take a picture of the observing site: show that it is flat.

\subsection{Describe the Sky}

What part of the sky can you see from this area?

If you have a lot of trees or buildings around, you may be limited in
the areas on the sky you can observe!  Specify the visible area of the sky in terms of altitude and azimuth ranges.
For example, you might say, ``I can see from 40 degrees up to 90 degrees in altitude if I look straight south, but
I can only see from 80 to 90 degrees in altitude if I look north because the neighbor's roof blocks the way.''

Using a sky-viewing app (google sky, etc), determine what is visible at the time you are doing the lab.
Visible means: not obstructed by buildings or by trees.  If there are a few leaves in the way, it
might be OK, but if there's a thick tree canopy above, it will not.
Record the time and a list of celestial objects (constellations, interesting features in those constellations)
that could be seen from the site.

Remember that radio observations can be performed during the day.  However, you can't see through the sun,
and for our telescopes, you need to be pointed $\gtrsim10\deg$ away from the sun to avoid contamination.

You will also need to point at the sun for part of the assignment, so figure out when the sun is directly
visible from your observing site.

\subsubsection{RFI sources}
Are there any other notable obstructions you might worry about?  For example, is there a cell tower
or a radio broadcasting center that might muck up your observations with radio frequency interference (RFI)?
Are there overhanging powerlines?  Remember that radio receivers can pick up signals from all around - 
if there are broadcasting devices (like car radios and cell phones), they can produce interference.
Note what possible sources are nearby.


\subsubsection{Describe the sky \emph{at other times}}
Using your observation planning tools, determine what else is observable from the site.
If you picked a really clear site, the answer might be `nearly everything' - but it's definitely not \emph{everything},
so summarize concisely what \emph{is} visible.

\emph{Also} answer these specific questions:
\begin{itemize}
    \item At what times does the Galactic Plane pass overhead?  
    \item What part of the Galactic Plane (what Galactic Longitude) passes over
        at those times?
\end{itemize}


\section{Rubric}
This assignment will be graded on the following criteria:
\begin{itemize}
    \item Did you turn in a complete writeup?
    \item Did you include photographs of the site and surroundings?
    \item Did you determine what is observable from the site?
    \item Did you address sources of interference?
    \item Did you address the infrastructure needs?
\end{itemize}

\end{document}
