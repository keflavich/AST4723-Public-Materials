\documentclass[11pt]{article}
%\usepackage{geometry}
\usepackage[inner=1.5cm,outer=1.5cm,top=2.5cm,bottom=2.5cm]{geometry}
\pagestyle{empty}
\usepackage{graphicx}
\usepackage{fancyhdr, lastpage, bbding, pmboxdraw}
\usepackage[usenames,dvipsnames]{color}
\usepackage{indentfirst}
\definecolor{darkblue}{rgb}{0,0,.6}
\definecolor{darkred}{rgb}{.7,0,0}
\definecolor{darkgreen}{rgb}{0,.6,0}
\definecolor{red}{rgb}{.98,0,0}
\usepackage[colorlinks,pagebackref,pdfusetitle,urlcolor=darkblue,citecolor=darkblue,linkcolor=darkred,bookmarksnumbered,plainpages=false,pdflang=en-US,pdftitle=AST4723_2022_Spectroscopy]{hyperref}
\renewcommand{\thefootnote}{\fnsymbol{footnote}}

\pagestyle{fancyplain}
\fancyhf{}
\lhead{ \fancyplain{}{AST4723} }
%\chead{ \fancyplain{}{} }
\rhead{ \fancyplain{}{\today} }
%\rfoot{\fancyplain{}{page \thepage\ of \pageref{LastPage}}}
\fancyfoot[RO, LE] {page \thepage\ of \pageref{LastPage} }
\thispagestyle{plain}

%%%%%%%%%%%% LISTING %%%
\usepackage{listings}
\usepackage{caption}
\DeclareCaptionFont{white}{\color{white}}
\DeclareCaptionFormat{listing}{\colorbox{gray}{\parbox{\textwidth}{#1#2#3}}}
\captionsetup[lstlisting]{format=listing,labelfont=white,textfont=white}
\usepackage{verbatim} % used to display code
\usepackage{fancyvrb}
\usepackage{acronym}
\usepackage{amsthm}
\VerbatimFootnotes % Required, otherwise verbatim does not work in footnotes!

\def\todo#1{\textcolor{red}{#1}}


\definecolor{OliveGreen}{cmyk}{0.64,0,0.95,0.40}
\definecolor{CadetBlue}{cmyk}{0.62,0.57,0.23,0}
\definecolor{lightlightgray}{gray}{0.93}



\lstset{
%language=bash,                          % Code langugage
basicstyle=\ttfamily,                   % Code font, Examples: \footnotesize, \ttfamily
keywordstyle=\color{OliveGreen},        % Keywords font ('*' = uppercase)
commentstyle=\color{gray},              % Comments font
numbers=left,                           % Line nums position
numberstyle=\tiny,                      % Line-numbers fonts
stepnumber=1,                           % Step between two line-numbers
numbersep=5pt,                          % How far are line-numbers from code
backgroundcolor=\color{lightlightgray}, % Choose background color
frame=none,                             % A frame around the code
tabsize=2,                              % Default tab size
captionpos=t,                           % Caption-position = bottom
breaklines=true,                        % Automatic line breaking?
breakatwhitespace=false,                % Automatic breaks only at whitespace?
showspaces=false,                       % Dont make spaces visible
showtabs=false,                         % Dont make tabls visible
columns=flexible,                       % Column format
morekeywords={__global__, __device__},  % CUDA specific keywords
}

\begin{document}

\title{Spectroscopy Observing Plan}


\section{Assignment}
Plan a spectroscopic observing run.  

 

You will observe:

\begin{itemize}
 \item A binary system with separation between 15" and 60" consisting of two medium-bright stars ($m_v < 10$)
 \item A bright ($m_v < 6$) A-star
 \item A bright ($m_v < 6$) M-star
 \item A spectrophotometric standard star with $m_v < 12$
 \item A bright nebula
 \item A dense star cluster
 \item A planet (and its moons if possible)
\end{itemize}
 

Your plan will include:

\begin{itemize}
    \item [5 pts] Magnitude estimates in the blue, visual, and red
        filters\footnote{There are many different filter sets that include BVR
        filters, including Johnson, Cousins, Bessel, Strömgren.  Any are
        acceptable for this estimate.}
    \item [10 pts] Optical finder charts with a field of view of 15'.
        For the print version, use a white background with black stars.
         Remember this can be quite tricky for our selection of bright
        sources; if the DSS doesn't have a good image, we will need to search
        for alternatives;
        \url{https://nova.astrometry.net/user_images/location}  may be a good option
        \emph{You need this finder chart accessible in both digital and print form:
        you will be using this document during your observing, so the quality of the 
        print will matter.}
    \item [5 pts] Contingency plans for observations in September, October, or
        November.  (Plan for September, but say what changes if you observe in
        late November instead).  Plan to observe from sunset to ~3.5 hours
        after sunset.
    \item [5 pts] A planned list of exposures and \emph{when} you will take them.
        These should be precise to the minute.  What CCD/CMOS images do you
        need?
\end{itemize}
 

 

\section{Reference Materials}
Spectrophotometric standard star lists

\begin{itemize}
    \item \url{https://www.eso.org/sci/observing/tools/standards/spectra.html}
    \item \url{https://www.eso.org/sci/observing/tools/standards/spectra/wdstandards.html}
    \item \url{http://mingus.mmto.arizona.edu/~bjw/mmt/spectro_standards.html}
    \item \url{https://www.naoj.org/Observing/Instruments/FOCAS/Detail/UsersGuide/Observing/StandardStar/Spec/SpecStandard.html}
    \item \url{https://noirlab.edu/science/observing-noirlab/observing-ctio/Spectrophotometric-Standards}
\end{itemize}
 

Binary star lists:

\begin{itemize}
    \item \url{https://www.astroleague.org/files/u220/DS-MasterObjectList2021.pdf}
    \item \url{http://www.ianridpath.com/binaries.html}
\end{itemize}

\section{Turn In:}
\begin{itemize}
    \item Your observing plan
    \item Your complete work notebook in .ipynb and .pdf form (assuming you used a notebook; you can complete the above without)
\end{itemize}


\end{document}
% resource: https://www.youtube.com/watch?v=_eMNRa-KEiQ
