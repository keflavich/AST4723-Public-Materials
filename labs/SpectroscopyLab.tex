\documentclass[11pt]{article}
%\usepackage{geometry}
\usepackage[inner=1.5cm,outer=1.5cm,top=2.5cm,bottom=2.5cm]{geometry}
\pagestyle{empty}
\usepackage{graphicx}
\usepackage{fancyhdr, lastpage, bbding, pmboxdraw}
\usepackage[usenames,dvipsnames]{color}
\usepackage{indentfirst}
\definecolor{darkblue}{rgb}{0,0,.6}
\definecolor{darkred}{rgb}{.7,0,0}
\definecolor{darkgreen}{rgb}{0,.6,0}
\definecolor{red}{rgb}{.98,0,0}
\usepackage[colorlinks,pagebackref,pdfusetitle,urlcolor=darkblue,citecolor=darkblue,linkcolor=darkred,bookmarksnumbered,plainpages=false,pdflang=en-US,pdftitle=AST4723_2024_SpectroscopyLab]{hyperref}
\renewcommand{\thefootnote}{\fnsymbol{footnote}}

\pagestyle{fancyplain}
\fancyhf{}
\lhead{ \fancyplain{}{AST4723} }
%\chead{ \fancyplain{}{} }
\rhead{ \fancyplain{}{\today} }
%\rfoot{\fancyplain{}{page \thepage\ of \pageref{LastPage}}}
\fancyfoot[RO, LE] {page \thepage\ of \pageref{LastPage} }
\thispagestyle{plain}

%%%%%%%%%%%% LISTING %%%
\usepackage{listings}
\usepackage{caption}
\DeclareCaptionFont{white}{\color{white}}
\DeclareCaptionFormat{listing}{\colorbox{gray}{\parbox{\textwidth}{#1#2#3}}}
\captionsetup[lstlisting]{format=listing,labelfont=white,textfont=white}
\usepackage{verbatim} % used to display code
\usepackage{fancyvrb}
\usepackage{acronym}
\usepackage{amsthm}
\VerbatimFootnotes % Required, otherwise verbatim does not work in footnotes!

\def\todo#1{\textcolor{red}{#1}}


\definecolor{OliveGreen}{cmyk}{0.64,0,0.95,0.40}
\definecolor{CadetBlue}{cmyk}{0.62,0.57,0.23,0}
\definecolor{lightlightgray}{gray}{0.93}



\lstset{
%language=bash,                          % Code langugage
basicstyle=\ttfamily,                   % Code font, Examples: \footnotesize, \ttfamily
keywordstyle=\color{OliveGreen},        % Keywords font ('*' = uppercase)
commentstyle=\color{gray},              % Comments font
numbers=left,                           % Line nums position
numberstyle=\tiny,                      % Line-numbers fonts
stepnumber=1,                           % Step between two line-numbers
numbersep=5pt,                          % How far are line-numbers from code
backgroundcolor=\color{lightlightgray}, % Choose background color
frame=none,                             % A frame around the code
tabsize=2,                              % Default tab size
captionpos=t,                           % Caption-position = bottom
breaklines=true,                        % Automatic line breaking?
breakatwhitespace=false,                % Automatic breaks only at whitespace?
showspaces=false,                       % Dont make spaces visible
showtabs=false,                         % Dont make tabls visible
columns=flexible,                       % Column format
morekeywords={__global__, __device__},  % CUDA specific keywords
}

\begin{document}

\title{Spectroscopy Lab: Introduction to `long-slit' spectroscopy}
\maketitle

In this lab, you will practice using a spectrograph and reducing `long-slit'
spectroscopy data.
% We are actually using a fiber-fed spectrograph instead of a long-slit spectrograph,
% so we have created a `pseudo-slit' by laying out the fibers in a linear pattern.

Your learning goals for this lab are:

\begin{itemize}
    \item to obtain spectra of astronomical sources
    \item to calibrate and reduce spectroscopic data
    \item to measure spectral lines
\end{itemize}

The lab includes observation, data analysis, and data reduction.


% There will be no formal writeup for this lab.  You will reduce the data and answer questions about it.

%Download the data from \url{https://ufl.instructure.com/files/54125722/download?download_frd=1}

\section{Observations}

We observed the following targets:

\begin{itemize}
    \item Jupiter
    \item A moon of Jupiter
    \item The Ring Nebula
    \item Vega
    \item At least one binary star system
\end{itemize}

We also observed calibration lamps.  The Neon lamp spectrum should serve as the basis of your calibration.

\section{Reduction}
\textbf{Recommendation: Do the full process described here (reduction, trace
and extract, wavelength calibrate) for \emph{one} star (of your choice) before
doing any of the other stars.}


\begin{enumerate}
    \item \textbf{Reduce the data by dark-subtracting them.}
Dark-subtract each of the different exposure types by matching the exposure times listed in the file names and confirmed in the FITS headers.

\item \textbf{Median-combine the images of the same source to reject cosmic rays.}
    However, if there is offset between the images that would result in losing signal, do not combine them.
\begin{itemize}
    \item \textbf{Only median-combine images if they are already aligned.}
    \item \textbf{Do not attempt to median-combine images at different flux levels.}
    \item \textbf{If you can't median-combine in the image, you can median-combine the normalized spectra instead.}
    \item \textbf{Not all of the spectra are ``good".  If an exposure has weird artifacts, note that and exclude it from further analysis (your note should include an image showing what the problem is)}
\end{itemize}

    \item \textbf{Trace, Wavelength Calibrate, Extract, and Flux Calibrate:} For the spectroscopic reduction, follow the process described in the
Spectroscopy lectures and on the astropy guides:

\begin{itemize}
    \item \url{https://learn.astropy.org/tutorials/1-SpectroscopicTraceTutorial.html}
    \item \url{https://learn.astropy.org/tutorials/2-WavelengthCalibration.html}
    \item \url{https://learn.astropy.org/tutorials/3-Trace_Extract_Wavelength-CalibrateSpectrum.html}
\end{itemize}

\end{enumerate}

Your aim is to produce wavelength-calibrated one-dimensional spectra for each of the targets.

% A modified version of the tracing guide was given in class to account for our slightly more poorly-behaved spectra.




\section{Trace and Extract}
Trace and extract spectra from each of the targets that has a continuous spectrum.
Use either the combined (median combined) spectrum or the brightest spectrum among the exposures you obtained.

Use the traces from each of these targets to obtain the corresponding extracted
spectrum from your combined flat field image. This process is shown in the
Wavelength Calibration notebook.  In brief: obtain the trace and trace profile
of the star, which give you the (x,y) locations to pull the spectrum from, then
use that to extract the spectrum just like we did for the mercury, neon, and
krypton lamps.

Use a trace from one of the stars (probably the brightest) to extract a spectrum from Jupiter.
{Our Jupiter spectrum images (2D spectra) consist of one spectrum of an `extended' object, just like the
calibration lamps that fill the spectrograph. To obtain a single spectrum with the appropriate
wavelength and flux calibration, use the trace of a single star assuming
that the star's spectrum has the same curvature as the extended source's 
spectra.}

Use the traces from each of the pointlike targets to obtain the corresponding extracted
spectrum from the wavelength calibration spectra (He, Ne).
Assess whether the wavelength solutions will be different for each extracted
spectrum: do the lines appear at different pixel positions?

{Note that there are calibration lamp spectra from the beginning and end of the night.
 For each spectrum you're extracting, check how much the calibration
 lamp spectra differed from the beginning to the end of the night: how
 much did the wavelength solution change?  What does that imply for your
 uncertainty on the wavelength solution?}

\section{Wavelength Calibrate}

Wavelength calibrate your spectra.

Show your work!

Estimate the uncertainty in your wavelength solution.

{Uncertainty estimation should follow the wavelength-calibration notebook,
        but you do not need to prove that the uncertainty is dominated by systematic errors, you
        can just estimate the standard deviation of the fit residuals.}

{I hope you can just repeat exactly what I did in the notebook without extensive
        fine-tuning of parameters.  If you find yourself getting lost, spending a lot of time here,
        ask for help and/or move on!  At least do the neon-based wavelength solution identification,
        since you need that for the next step, though.}





\section{Fit and identify spectral lines}

Once you have your wavelength-calibrated spectra, you will use Vega as a spectral flat field as decribed
in the ``Line Profile Measurement'' lecture.  As part of that process, you will fit line profiles to the
hydrogen lines in Vega's spectrum.

Fit the H$\beta$ 4861 angstrom line from any star in which it is detected.
Report the velocity of the line and the equivalent width.

{You need to select the part of the spectrum that contains this line and \emph{normalize} it, most likely by fitting the continuum with a linear model.}

{You should fit a Gaussian absorption line and use its integral to get the EQW.}

{You can fit these two together, following the example in the `measuring line properties' notebook.}



In Jupiter's spectrum, which is the reflected solar spectrum, identify and fit (and measure EQWs of) as many Fraunhofer lines
(\url{https://en.wikipedia.org/wiki/Fraunhofer_lines}) as you can.
Make a table of the measured wavelength, width, EQW, and uncertainty on
those fitted quantities.

In the twilight spectrum, you may also see these lines.  Are they at the same wavelength?  Are they the same relative depth?

Compare to your other stellar spectra.  Which lines are detected in other stars?  Are there lines detected in other stars not seen in the Sun's spectrum?
Again, report this in a table.

You should have one table for each target (each star, Jupiter and its moon(s), and the nebula).

\section{Measurements}

Your measurements should include:

\begin{itemize}
    \item The measured wavelength coverage, and pixel spacing, from your wavelength calibration solutions.
        (especially note: if you took multiple calibration spectra, did the wavelength solution change?)
    \item The Equivalent Width of the hydrogen lines in Vega's spectrum
    \item The wavelength of any emission lines in the Ring Nebula's spectrum
    \item The wavelength, and identity, of spectral lines in other targets (you may be limited to Hydrogen lines, but there should be some Fraunhofer lines identifiable too)
    \item The pixel scale of the slit viewer on the acquisition and guiding system (you will get this from the observations of the binary system)
\end{itemize}


\section{Turn In:}
\begin{itemize}
    \item Your formal writeup
    \item Your complete work notebook in .ipynb and .pdf form
    \item Plots of each of the wavelength-calibrated spectra
    \item Tables of measurements
\end{itemize}


The writeup rubric is:
\begin{enumerate}
    \item Introduction (5\%): Describe the purpose of the lab \& background
    \item Procedure (55\%): Describe the setup of the hardware and the data taking and reduction process
    \item Data Analysis (25\%):  Report and analyze your measurements
    \item Data Packaging and Delivery (10\%): Put the data together and turn them in with appropriate metadata
    \item Conclusions (5\%): Summarize what you learned
\end{enumerate}

\end{document}
%
% resource: https://www.youtube.com/watch?v=_eMNRa-KEiQ
